\begin{rubric}{Présentation}

\text{
Ingénieur mécanique formé à l'Université Nacional de Colombie, titulaire d'un Master II en Management de l'Innovation et du Design Industriel et PhD. en Génie des Systèmes Industriels de l'Université de Lorraine.

Mes expériences professionnelles et de recherches se focalisent sur le champ de la fabrication additive open source (également appelé Impression 3D)  comme un vecteur de développement industriel durable.
Un premier axe porte sur la validation de la qualité du procédé d'impression 3D open source en tant qu'outil reproductible pour la fabrication de pièces. 
Un deuxième axe est centrée sur la faisabilité technique du recyclage des polymères pour les processus d'impression. La caractérisation mécanique et chimique de la matière recyclé dans la chaîne d'impression a été étudié en co-tutelle avec l'Équipe de Recherche sur le  Processus Innovatifs (ERPI) et le Laboratoire des Réactions et Génie des Procédés (LRGP --- UMR 7274) à Nancy.

Je travaille actuellement sur la création d'un démonstrateur local de l'approche de recyclage distribué et circuit court dans le cadre d'un projet EU H2020 afin d'inscrire cette technologie dans une logique liée aux enjeux de l'économie circulaire.  De plus, je participe à un projet EU Erasmus+ pour mise en place de laboratoires d'innovation sociale liées à des thématiques de changement climatique en Amérique du sud.

Je suis intéresse sur le développement des scénarios systémiques sur les enjeux  liés au croisement de la technologie de la fabrication additive et le développement durable.
}

% \entry*[\textbf{Fonction}:]

% 	Chercheur contractuel au sein de la plateforme Lorraine Fab living Lab (LF2L) du  laboratoire ERPI (Équipe de Recherche sur les Processus Innovatifs) de l'Université de Lorraine.
   
%     Participation au projet H2020 EU INEDIT (open INnovation Ecosystems for Do It Together process) dont l'objectif principal est de faire évoluer l’approche do-it-youself (DIY) caractérisant notamment les FabLabs au profit des PMEs.

\end{rubric}
