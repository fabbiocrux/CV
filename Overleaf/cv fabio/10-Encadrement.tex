\begin{rubric}{Activités d'encadrements pédagogique}
\subrubric{Projets Industriels}
\entry*[2017-2018] 

Project Holipresse 1AI: Création d'un moule par système de strato-conception low-cost pour la fabrication de pièces injectés à partir de bouchons de bouteilles recyclés. 

\entry*[2019-2020] 
Plast'If (2019- 2020): Test de caractérisation de matière plastique récyclée pour l'utilisation des machine d'impression directe -Fused Granular Fabrication FGF-.

\subrubric{Projets Pédagogiques}

\entry*[2017-2018] 
Project Fabcity Nancy 2AI: Cartographie des initiatives sur de démarches de: 
Plus de détails: \url{http://fabcity-nancy.fr/}

\entry*[] 
Projet Etudiant ERASMUS 2AI : Création de manuels en plusieurs langues pour faciliter l'utilisation des machines au sein du Lorraine Fab Living Lab.

\subrubric{Projet de micro-thèse}
\entry*[] 
Accompagnement des étudiants (10) pendant 5 semaines dans le module "Initiation de Recherche" de l'écolé d'Ingénieurs CESI Nancy. 
Création de prototypes pour l'extrusion de filament en utilisant open hardware.
Plus de détails dans le lien:
\url{http://lf2l.fr/projects/green-fablab/} 

\subrubric{Projets  Master Recherche}

\text{
Participation avec les professeurs de l'ENSGSI à l'encadrement des étudiants en master recherche et travail de thèse:
}

\entry*[2018- présent]
Participation aux travaux de thèse de Pavlo SANTANDER (2018-présent) concernant l'identification et la sélection de scénarios prospectifs les plus convenables au niveau \textit{supply chain} du recyclage de polymères en utilisant des méthodes mathématiques d'optimisation.
Ma contribution porte sur la validation des paramètres techniques nécessaires pour l'adéquation du modèle mathématique en utilisant le cadre expérimental développé dans le Green Fablab au sein du Lorraine Fab living lab.\newline

\textit{\textbf{Production scientifique associée:}} Un article en revue international et une conférence internationale concernant ce travail de recherche ont été publiés (2ème Auteur).

\entry*[2017 - 2018]
Participation au travail de Ramsés MOREIRA de ALBUQUERQUE sur la création de \textit{dry models} pour l'apprentissage des tâches complexes nécessaires à la chirurgie d'un étudiant d'odontologie.
L'objectif principal est donc d'utiliser les technologies d'impression 3D disponibles dans le Lorraine Fab Living Lab (LF2L) dans le but fournir et évaluer un modèle imprimé pour former des nouveaux étudiants en odontologie.
Dans ce contexte, le laboratoire ERPI a travaillé avec l'école de Chirurgie de Nancy.
Ce travail est également en lien avec le projet des étudiants en M2 BSIS - Ingénierie Biomédicale de l'Écolé de chirurgie.\newline
Ma contribution porte sur l'aide dans la fabrication des échantillons et l'adéquation des paramétres d'impression pour les tests nécessaires.\newline
\textit{\textbf{Production scientifique associée:}} Une communication concernant ce travail de recherche a été publiée sur une conférence internationale (3ème Auteur).

\entry*[2018-2019]
Participation au travail de Arthur Lucas GRANGEIRO sur  la  caractérisation de machine d'impression directe (Fused Granular Fabrication).\newline
\textit{\textbf{Production scientifique associée:}} En cours de soumission.

\entry*[2018-2019]
Participation au travail d'Anamaria BARRERA BOGOYA Analyse des facteur sociaux pour le recyclage distribué. Travail en co-tutelle avec l'Université de Freigburg.\newline
\textit{\textbf{Production scientifique associée}} En cours de rédaction.



\subrubric{Projet d'enseignement}
\text{
L'expérience de création de cours pour les professeurs de collèges et des lycées m'a conduit à concevoir un module qui pourrait être proposé à des élèves ingénieurs ou de master afin de se former aux techniques de prototypage et de fabrication additive tout en étant respectueux de l'environnement.
Ce projet d'enseignement pourrait s'intituler \textbf{"Recyclage de matière plastique pour impression 3D: les atouts de la collecte en circuit court"}

L’objectif est de proposer un cheminement cohérent et progressif aux étudiants en partant de l'analyse des besoins, co-créations de solutions, et prototypage des objets de conception intermédiaire à l'aide de techniques de l'impression 3D et en identifiant une réutilisation possible de gisement aujourd'hui non valorisables.
Ce module aurait pour objectif de combiner les approches open hardware et \textit{Faire-soi-même} afin d'eco-concevoir des produits et des procédés qui répondent aujourd'hui à la stratégie des enjeux de l'économie circulaire.

}

\end{rubric}
