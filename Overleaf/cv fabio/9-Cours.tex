\begin{rubric}{Activités d'enseignement}


\entry*[]\textbf{Pôle Conception et Innovation Module Ingénierie de l’innovation II / Design Thinking}

Cours d'introduction sur la méthode de conception "Design Thinking".  Cette approche de l'innovation s'appuie sur un processus de co-création impliquant des retours de l'utilisateur final.  Elle permet de développer un produit ou un service innovant qui soit à la fois désirable, viable et faisable par la combinaison des approches humaines, économiques et technologiques.

 \textit{Ma contribution:} Conception et animations des séances de TD pour la partie prototypage.

\entry*[]\textbf{Module CI14. Introduction à la recherche}

Ce cours permet aux étudiants  de maîtriser les  fondamentaux sur le processus de recherche scientifique et faire un travail de synthèse sous la forme de rédaction d'un article d'état de l'art. 
Connaître les base des données scientifiques, presentation de la structure des articles scientifiques et la gestion de référence (Mendeley et Zotero). 
  
\textit{\textbf{Ma contribution:}}: Animations de séances TD sur l'application de la méthode de revue systématique de la littérature  à partir d'une équation ciblée. 
  


\entry*[]\textbf{Summer school «Neuromarketing and Innovation»}
Ce module a été dispensé pour des étudiants et des industriels de la région de Costa Rica. 
La thématique a concerné l'identification des étapes dans un processus de création des projets d'innovation,  l'évaluation prospective du degre de nouveauté en utilisation des technologies eye-tracking et neuro-lab (e.g. capteurs physiologiques)

\textit{\textbf{Ma contribution:}} Cours introductif sur la creation de prototypes en tant que support pour l'évaluation de l'acceptabilité avec des eye-tracking. Accompagnement des groupes à la création d'un objet marketing promotionnelle de Costa Rica afin d'être evalué sous ce technologie. 
  

\entry*[]\textbf{Summer School collegium LMI/Lorraine INP "From Idea – to market"}

Ce module a été donné aux étudiants qui ont une idée  d'un projet entrepreneurial ou start-up. 
Le summer school a été en collaboration avec le pôle entreprenarial de nancy.
\textit{\textbf{Ma contribution:}} Presentation de l'impression 3D comme technologie d'aide à la creation et decision pour  tester des concepts/solutions. 


\entry*[]\textbf{Introduction au prototypage et à l’impression 3D}

Ce module a été dispensé auprès des public de professeurs de technologie la Académie de Nancy-Metz. Il a donc été nécessaire d'en adapter les contenu et le format en fonction des connaissances initiales des interlocuteurs. 
Le sujet et discussion ont portée sur la pris en main des étapes pour le processus d'impression 3D, depuis model numerique CAO jusqu'à la determination des paramétres en fonction de la matières utilisé.
Cette expérience a été très constructive d'un point de vue pédagogique pourvu des l'implementation des cette type de technologie dans le collegès et lycèes de la region. 

\end{rubric}

